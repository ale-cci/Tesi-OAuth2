\section{Json Web Token}
\label{jwt}
\acs{jwt} (\acl{jwt}) is an internet standard for creating \acs{json} based tokens.
It asserts some number of claims in the token itself, so that authentication
protocols that uses it are completely stateless.
In other words, if i give a JWT to a completely different service, it is
completely able to verify and authenticate the incoming request, without having
to perform a query on a database.

The key concept that holds together the JWT standard is encryption, that could
be either symmetric or asymmetric.

JWTs could be either JWS, if they have been signed with an asymmetric key, or JWE
if they have been encrypted with a symmetric key.

In a zero trust architecture, symmetric encryption is rarely used: if a
single server with access to the encryption key is compromised, then all the
JWTs created with that specific key are no longer to be trusted, and should be revoked
in the whole architecture.

For this specific reason, asymmetric encryption is preferred most of the times.
Therefore the undersigned has made the arguable decision to not include the JWE
specification in this dissertation.

\subsection{JWT Structure (JWS)}
\begin{lstlisting}
jwt = access_token
chunks = jwt.split('.')

if len(chunks) == 5:
    raise ValueError('Hey this is a JWE!')

elif len(chunks) != 3:
    raise ValueError('Malformed JWT!')

head, body, signature = chunks
\end{lstlisting}

As seen before, JWT chunks are glued together using a dot character.
Immediately looking at the number of chunks we could differentiate JWS and JWE,
since the latter has only five of them, and the former only three.

Since we have already seen how a JWT body could look like, let's keep using the example JWT
in figure \ref{fig:id_token} for analyzing the head and the signature.

\subsubsection{JWT head}
\begin{lstlisting}
>>> json.loads(head_decoded)
{'alg': 'RS256', 'kid': '1', 'typ': 'JWT'}
\end{lstlisting}
The JSON fields are pretty straightforward:
\begin{itemize}
    \item \texttt{alg} : The algorithm used to sign the JWT
    \item \texttt{kid} : Key ID, which indicates which key has been used to sign the JWT
    \item \texttt{typ} : What type of object is this (spoiler: a JWT)
\end{itemize}

The most important field among the ones listed above is \texttt{kid}.
With that, the client application could immediately locate which public key should use to verify the
JWT signature.

Keys used to sign or encrypt a JWT are called JWK (Json Web Keys).
They could be stored in the client application filesystem, or more conveniently
be exposed by the with a public endpoint by the authentication server (see public key sharing at page \pageref{jwks}).

\subsubsection{JWS signature}
Signatures are a really important features on JWT, they grant you the
confidence to say: \textit{This JWT has been created by the authentication
server, and no one has tampered with it}.

The signature is calculated on the head and body, encoded and joined with the
dot character.

Different asymmetric signing algorithms could be used for calculating the
signature of the token, but in the end, the used algorithm is reported in the
JWT's head.

\begin{lstlisting}[caption=Calculate JWT checksum]
import base64
from cryptography.hazmat.primitives import serialization
from cryptography.hazmat.primitives.asymmetric import padding
from cryptography.hazmat.primitives import hashes

def checksum(message, key_path):
    with open(keypath, "rb") as key_file:
        private_key = serialization.load_ssh_private_key(
            key_file.read(),
            None,
            default_backend()
        )

    # Sign a message using the key
    signature = private_key.sign(
        message,
        padding=padding.PKCS1v15(),
        algorithm=hashes.SHA256()
    )

    return base64.b64encode(signature).rstrip(b'=')

if __name__ == '__main__':
    # assume head and body already defined.
    payload = '.'.join([head, body])
    checksum(payload, '~/.ssh/id_rsa')
\end{lstlisting}

\subsection{Sharing public keys}
\label{jwks}
To avoid copy-pasting public keys between multiple services, usually the authentication
server exposes them in the \texttt{.well-known/jwks.json} endpoint, using the
\href{https://tools.ietf.org/html/rfc7517#section-4}{JWK Format}.

Here is an example:

\begin{lstlisting}[language=bash]
~: curl -s http://localhost:4000/oauth/.well-known/jwks.json \
        | python -m json.tool
{
    "keys": [
        {
            "alg": "RS256",
            "kid": "1",
            "kty": "RSA",
            "n": "0wMtv3UjBSWh[...]niyhgsO",
            "e": "AQAB",
            "use": "sig"
        }
    ]
}
\end{lstlisting}
With a quick look to the json returned by the curl response, it is obvious that
more than one key could be returned using the adopted formalism.

Each entry in the \texttt{code} array is required to have the keys:
\begin{itemize}
    \item \texttt{kty} : (key type) identifies the cryptographic family algoritm
        used with with the key. Possible values could be RSA or EC.
        The value is a case sensitive string, and MUST be present in a JWK.
    \item \texttt{use} : (public key use), identifies the intended use of the
        public key. The use is employed to indicate whether a public key is used
        for encrypting data or verifying the signature. Values defined by the
        specification are: \texttt{sig} or \texttt{enc}.
        The use is is optional, unless the application requires it.
    \item \texttt{alg}
    \item \texttt{e} and \texttt{n}: public base and exponent
    \item \acs{kid}: \acl{kid}
\end{itemize}

For more informations about the

JSON Web Keys are javascript object notation data structures that represents a
cryptographic key.

\subsection{JWE}
\acs{jwe} (\acl{jwe}) differences from the previous \acs{jws} (\acl{jws}),
defines a way to encrypt your claims, so that only the intended
receiver could read the information presented by the token.

The main advantage that it has over JWS is the protection against \acs{mitm} attacks,
since the attacker cannot see the information contained in the token unless he
has access to the encryption key.
