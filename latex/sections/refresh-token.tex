\section{Refresh token}
\begin{figure}[h]
    \centering
    \begin{BVerbatim}
+--------+                                           +---------------+
|        |--(A)------- Authorization Grant --------->|               |
|        |                                           |               |
|        |<-(B)----------- Access Token -------------|               |
|        |               & Refresh Token             |               |
|        |                                           |               |
|        |                            +----------+   |               |
|        |--(C)---- Access Token ---->|          |   |               |
|        |                            |          |   |               |
|        |<-(D)- Protected Resource --| Resource |   | Authorization |
| Client |                            |  Server  |   |     Server    |
|        |--(E)---- Access Token ---->|          |   |               |
|        |                            |          |   |               |
|        |<-(F)- Invalid Token Error -|          |   |               |
|        |                            +----------+   |               |
|        |                                           |               |
|        |--(G)----------- Refresh Token ----------->|               |
|        |                                           |               |
|        |<-(H)----------- Access Token -------------|               |
+--------+           & Optional Refresh Token        +---------------+
    \end{BVerbatim}
    \caption{Refresh token flow}
\end{figure}

Refresh tokens are credentials used to obtain access tokens. Refresh tokens are
issued to the client by the authorization server and are used to obtain a new
access token when the current access token becomes invalid or expires, or to
obtain an additional access token identical or narrower scope.

Issuing a refresh token is completely optional, at the desecration of the
authorization server.
If the authorization server issues an access token, it is included with the
access token (Step B).

The value of the refresh token is usually a string opaque to the client, unlike
access token, it is only intended to be used by the authorization server,
therefore this kind of token is not sent to the resource server when accessing
protected resources.


