\section{Refresh token}
% TODO: explain why you need access tokens.
For now all fine and dandy, but what if you need to perform
another request after an hour or maybe the next day?

Access tokens lifetime is really short, usually in the range from fifteen
minutes to an hour, ad for sure you don't want users of your application
to click the "grant permissions" button every single day, or worse every hour.

Fortunately refresh tokens come at our help!
They could be used for, (you guessed it) obtain a new access token, once the one
in use becomes invalid or expired.

They could also be used to obtain additional access token, with identical or
narrower scope.

Differently from access tokens, refresh tokens have a long lifetime (i.e. 6
months) or no expiry at all, and
are only intended to be used by the authorization
server.
\\
Therefore the refresh token is not sent to resource server alongside the access token
once a request to a private resource is performed.

\subsection{When does the client obtain the refresh token?}

\begin{figure}[h]
    \centering
    \begin{BVerbatim}
+--------+                                           +---------------+
|        |--(A)------- Authorization Grant --------->|               |
|        |                                           |               |
|        |<-(B)----------- Access Token -------------|               |
|        |               & Refresh Token             |               |
|        |                                           |               |
|        |                            +----------+   |               |
|        |--(C)---- Access Token ---->|          |   |               |
|        |                            |          |   |               |
|        |<-(D)- Protected Resource --| Resource |   | Authorization |
| Client |                            |  Server  |   |     Server    |
|        |--(E)---- Access Token ---->|          |   |               |
|        |                            |          |   |               |
|        |<-(F)- Invalid Token Error -|          |   |               |
|        |                            +----------+   |               |
|        |                                           |               |
|        |--(G)----------- Refresh Token ----------->|               |
|        |                                           |               |
|        |<-(H)----------- Access Token -------------|               |
+--------+           & Optional Refresh Token        +---------------+
    \end{BVerbatim}
    \caption{Refresh token in OAuth2}
    \label{fig:refresh-token-flow}
\end{figure}

Refresh tokens are issued to the client by the authorization server, and they
take the form of a string, opaque to the client application.

Issuing a refresh token is completely optional, at the desecration of the
authorization server.
If the authorization server issues a refresh token, it is included alongside the
access token (Step B of figure \ref{fig:refresh-token-flow}).


