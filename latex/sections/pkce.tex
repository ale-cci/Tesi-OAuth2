
\section{Protocol Vulnerabilities}
\subsection{PKCE (Proof Key for Code Exchange)}
\label{pkce}
Public applications, such as desktop and mobile apps, are unable to securely
store credentials, in our case, the client secret.

For this particular reason the implicit flow has been created, so
that the step after the authorization that exchanges the code and the secret
with the authorization code is skipped, so the authorization code is immediately
returned.

This particular sequence of events opens up a vulnerability:
when the access token is returned by the authorization response, malicious apps
running alongside our OAuth2 client are able to read it, and therefore use it as
our application would.

For example, if your client is running in a browser, let's say firefox just to
pick one, any plugin you have installed has access to the current url, and
therefore to the access\_token.

\ac{pkce} is a superset feature for OAuth2, idealized for preventing this kind of
attack, called "authorization code interception".

OAuth2.1 specification will mandate that all implementation of the authorization
grant type use \ac{pkce}.

\begin{enumerate}
    \item the client generates a random string, and calculates the hash using a
        hashing function
    \item the client performs the normal oauth2 flow, providing along with the
        client secret, the random string hashed value.
    \item when the authorization server performs a successful redirect with the
        'code', the client exchanges with a post request, the code and the
        unhashed secret with the access token.
\end{enumerate}

This feature blocks "easy" oauth2 attacks, that could be performed simply by
reading the access token from the GET request.

