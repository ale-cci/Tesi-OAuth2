% vim: colorcolumn=80
\section{Introduction}
Back when the first API services were invented, everything was easier.
\\
If you wanted to grant private data access to a third party application,
the sole thing you had to do was provide it your username and password, and
presto!
The application could freely access your account without any restrictions!

But things, especially now, are not so simple anymore: there is no guarantee
that an application will keep your credentials securely stored, or even worse
use them for malicious purposes, by accessing more informations than the bare
necessities.

To prevent this kind of behavior, and that everyone finds a different solution
to solve the same problem, we have agreed to standard protocols.
\\
The one discussed in this dissertation takes the name of \textbf{OAuth2}.

It is worth noting that this is not the only protocol that solves this kind of problem.
Another one that is widely adopted by the industry is \textbf{SAML}\footnote{More on SAML: \url{https://en.wikipedia.org/wiki/Security_Assertion_Markup_Language}},
but discussing it is not the scope of this paper.


OAuth2 is often confused with API keys\footnote{More on API keys:
\url{https://en.wikipedia.org/wiki/Application\_programming\_interface\_key}}
%
for their similar use and terminology.
\\
A simple way to distinguish the two is
is that API keys are used when the owner of an application owns the data that
the application uses, OAuth2 is used when the two differs.

\subsection{Example of OAuth2 use-case}

Let's assume you go on a vacation with your friends, during that vacation you
take a lot of pictures and videos, and your friends do it too.
Each one of you uploads the pictures and videos he took on Google cloud, and
you have a shared folder that weights a few Giga.

At the end of the vacation you want to organize all the pictures and videos
automatically, by adding them labels and name them all in a consistent way, for
example using the place and the date of where the picture was created.

This sounds like a lot of work to do all by hand, but thankfully you find an
online application that does exactly what you need: by combining image
processing and the EXIF tags of each picture or video it can automatically
create labels and rename them. Neat! Now you can delegate all the work with that
application.

Conveniently that application has a button called "Access with Google drive"
that when pressed opens up the Google login page and asks you if you want to
grant access to the application.

You perform the login with Google, accept the permissions and magically the
application starts doing it's work, labeling and renaming images.

% ================================================================================
\subsection{The terminology}
What you have just read is a typical example of the OAuth2 flow, but before deep
diving into the implementation and specification of this protocol, let's clear
some terminology.

First and foremost the \textbf{resource owner}, or in other words the one who
actually owns the \textbf{protected resource} or resources.

\begin{itemize}
    \item The \textbf{protected resources}, some form of private data that is
        not publicly accessible, in the example's case the vacation's photo and
        videos.
    \item \textbf{resource owner}: the one who own the protected resource, in
        the example's case you (and your friends).
    \item \textbf{authorization server} the service that knows the resource
        owner's identity, or in other words an application where the resource
        owner has an account. In the example's case it's Google.
    \item the \textbf{client} or \textbf{third party application} is the
        applications that requires access to the private resource (photos and
        videos), or access to perform some actions on the private resource
        (rename and assign them some labels).
    \item the \textbf{resource server} is the one that has access to the private
        data, sometimes is the same as the authorization, but as we see in our
        example, it could be different.
        In the example's case it's Google drive \footnote{to be clear, Google
        has two different services: one for authorization and one for Google
        drive}.
\end{itemize}

\subsection{Authentication and Authorization}
Authorization and authentication my sound similar to one another, but they are
two distinct concepts, used in identity and access management:
authentication says who you are, authorization says what actions you can
perform.

OAuth2 is only an authorization protocol, which means that it's only use is for
granting access to protected resources to a client.

Later we could see that it could be extended and used as an authentication
provider with OpenID connect (page \pageref{openid}).
