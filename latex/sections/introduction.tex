% vim: colorcolumn=80
\section{Introduction}
Back when the first API services were invented, everything was easier.
\\
If you wanted to grant private data access to a third party application,
the sole thing you had to do was provide it your username and password, and
presto!
The application could freely access your account without any restrictions!

But things, especially now, are not so simple anymore: there is no guarantee
that an application will keep your credentials securely stored, or even worse
use them for malicious purposes, by accessing more informations than the bare
necessities.

To prevent this kind of behavior, and that everyone finds a different solution
to solve the same problem, we have agreed to standard protocols.
\\
The one discussed in this dissertation takes the name of \textbf{OAuth2}.

It is worth noting that this is not the only protocol that solves this kind of problem.
Another one that is widely adopted by the industry is \acs{saml}\footnote{More on \acs{saml}\url{https://en.wikipedia.org/wiki/Security_Assertion_Markup_Language}}, but discussing it is not the scope of this paper.


OAuth2 is often confused with API keys\footnote{More on API keys:
\url{https://en.wikipedia.org/wiki/Application\_programming\_interface\_key}}
%
for their similar use and terminology.
\\
A simple way to distinguish the two is
is that API keys are used when the owner of an application owns the data that
the application uses, OAuth2 is used when the two differs.

\subsection{Example of OAuth2 use-case}

Lets say that last summer you went on vacation with your friends, during which
you took a ton of pictures and videos, and your friend did too.
Each one of you did upload the pictures and videos he took on Google Drive, and
now you have a few GB of stuff, stored on a shared folder.

Now, with the current pandemic, you are locked at home, and looking back at the pictures, the first thing that comes to your mind is:
\textit{"It would be really nice if i could find a way to organize them, by adding some labels of where the picture was shoot"}.

Unfortunately this sounds like a lot of work, and surely you are not interested
to do it all by hand.  But thankfully, when you are about to close your laptop,
you find an online application that fits exactly your needs: by combining image
processing and the EXIF tags of each picture or video, it can automatically tag
them with appropriate labels.
\\
Neat!  Now you can delegate all that boring work to that application.

There's a single problem: in order to tag the images, that web application
requires access them, and the only apparent way do so is by uploading them via
HTTP.

Conveniently that application has a button called "Access with Google drive"
that when pressed opens up the Google login page and asks you if you want to
grant access to the application.

You perform the login with Google, accept the permissions and magically the
application starts doing it's work, labeling them one by one.

% ================================================================================
\subsection{The terminology}
What you have just read is a typical example of the OAuth2 flow:
you want to grant access to an application some data which it does \textbf{not} own,
and most importantly you did \textbf{not} give to it your Google Drive
password, as strangely as it may sound.

But before deep diving into the implementation and specification of this protocol, let's clear
some terminology.

First and foremost there is you, the \textbf{resource owner}, or in other words the one who
actually owns the \textbf{protected resource} (aka. your vacation's pictures and videos).
\\
The protected resource is some form of private data, that is not
publicly accessible.  The server who stores and the protected resources is
called the \textbf{resource server}, and in this schema is the only one who
should have access to them.

In order to access the private data, a client or (third party application) must
provide an \textbf{access token} to the resource server.

Access tokens are issued by an \textbf{authorization server}, that sometimes,
but not always, could be the same as the resource server.

In our example the two were different: the authorization server was
\textit{Google} and the resource server was \textit{Google Drive}.

\subsection{Difference between Authorization and Authentication}
Authorization and authentication my sound similar to one another, but they are
two distinct concepts, used in identity and access management:
authentication says \textbf{who} you are, authorization says \textbf{what}
actions you can perform.

A key is an authorization, which says that you can open a door. Your ID card is
an authentication, since it says who you are.
A badge works both as an authorization and authentication, since it could be
used to say who you are and open doors.

OAuth2 is only an extensible authorization framework, which means that it's
only use is for granting access to protected resources to a client.

Later (page \pageref{openid}) we will see that OpenID Connect exploits it's
extensibility, and grants it authentication capabilities.
