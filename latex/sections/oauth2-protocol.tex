
\section{OAuth2 Protocol}
OAuth2 (abbreviation of Open Authentication) is an authentication framework,
which enables applications to obtain limited access to protected resources,
that reside on an HTTP service.

It works by separating the concerns of resource provision and authentication to
two separate entities: the resource server and the authorization server.

Once the access is granted to a third party application by the authorization server,
an access token cryptographically signed is issued.

Since the token is signed or encrypted, the resource server, can blindly trust
the forged tokens, and checks if the required authorizations are contained in
the token.


\subsection{Protocol Overview}
\begin{figure}[h]
    \centering
    \begin{verbatim}
     +--------+                               +---------------+
     |        |--(A)- Authorization Request ->|   Resource    |
     |        |                               |     Owner     |
     |        |<-(B)-- Authorization Grant ---|               |
     |        |                               +---------------+
     |        |
     |        |                               +---------------+
     |        |--(C)-- Authorization Grant -->| Authorization |
     | Client |                               |     Server    |
     |        |<-(D)----- Access Token -------|               |
     |        |                               +---------------+
     |        |
     |        |                               +---------------+
     |        |--(E)----- Access Token ------>|    Resource   |
     |        |                               |     Server    |
     |        |<-(F)--- Protected Resource ---|               |
     +--------+                               +---------------+
    \end{verbatim}
\end{figure}
Above you could see the main flow of the OAuth2 protocol, the final objective is
always to obtain an access token, which could be used to perform authorized
requests.


 \begin{itemize}
     \item[(A)] The client request an authorization from the resource owner.
         The authorization request is performed directly to the resource owner,
         or preferably via the authorization server server, as intermediary.
     \item[(B)] The client receives an authorization grant, which is a
         credential representing the resource owner's authorization, expressed
         using one of four grant types defined in the specification or using an
         extension grant type.
     \item[(C)] The client requests the access token, by authenticating with the
         authorization server and presenting the received authorization grant.
     \item[(D)] The authorization server authenticates the client and validates
         the received authorization grant, if valid the access token is
         returned.
     \item[(E)] the client requests the protected resource from the resource
         server providing the access token.
     \item[(F)] the resource server validates the access token. If valid the
         requested resource is provided at the client.
 \end{itemize}

\subsection{Different grants for different clients}
% TODO: expose better
The specification differs between public and confidential client.

\subsection{Different grant type}
The specification supports different ways of obtaining the access token, called
\textbf{grant types}.
\begin{enumerate}
    \item access code
    \item implicit
    \item client credentials
    \item client password
\end{enumerate}

\subsection{Access Token}
Access tokens are in any shape or form, credentials, that client application use
to access resources protected by resource servers.
\\
They take the form of a string, that usually (but not always) the client cannot
interpret.

Each access token is unique, and resource servers could verify the validity and
look up the scopes, granted by the resource owner to the access token.

The OAuth2 specification does not constrain the access token format, saying
that, quote: \\
\textit{``Access tokens can have different formats, structures, and methods of
utilization''}.

As we will see later, the OpenID Connect framework takes advantage of this
freedom, by imposing a strict token format (see \ac{jwt} at page \pageref{jwt}).

\subsection{Refresh token}
\begin{alltt}
  +--------+                                           +---------------+
  |        |--(A)------- Authorization Grant --------->|               |
  |        |                                           |               |
  |        |<-(B)----------- Access Token -------------|               |
  |        |               & Refresh Token             |               |
  |        |                                           |               |
  |        |                            +----------+   |               |
  |        |--(C)---- Access Token ---->|          |   |               |
  |        |                            |          |   |               |
  |        |<-(D)- Protected Resource --| Resource |   | Authorization |
  | Client |                            |  Server  |   |     Server    |
  |        |--(E)---- Access Token ---->|          |   |               |
  |        |                            |          |   |               |
  |        |<-(F)- Invalid Token Error -|          |   |               |
  |        |                            +----------+   |               |
  |        |                                           |               |
  |        |--(G)----------- Refresh Token ----------->|               |
  |        |                                           |               |
  |        |<-(H)----------- Access Token -------------|               |
  +--------+           & Optional Refresh Token        +---------------+
\end{alltt}

Refresh tokens are credentials used to obtain access tokens. Refresh tokens are
issued to the client by the authorization server and are used to obtain a new
access token when the current access token becomes invalid or expires, or to
obtain an additional access token identical or narrower scope.

Issuing a refresh token is completely optional, at the desecration of the
authorization server.
If the authorization server issues an access token, it is included with the
access token (Step B).

The value of the refresh token is usually a string opaque to the client, unlike
access token, it is only intended to be used by the authorization server,
therefore this kind of token is not sent to the resource server when accessing
protected resources.


The protocol is heavily dependent on HTTPS and HTTP redirects.

Before initiating the protocol, the client registers with the authorization
server.
The means through which the client registers are beyond the scope of the
specification, but typically involve end-user interaction with an HTML
registration form.

In the client registration, it is required to specify:
\begin{itemize}
    \item the client type
    \item it client redirection(s)
    \item include any other information required by the authorization server,
        for example the application name, logo website, description and others.
\end{itemize}

It starts with a specifically crafted URL, when the client needs the access to
the protected resource, it redirects the user to that URL.

The URL contains a query string
It starts with a specifically crafted redirect URL, that the clients
redirects the user to.
The URL points to the authorization's server, and it contains parameters in the
form of a query string. Here is an example:

\begin{lstlisting}
import urllib.request

callback_url = urllib.request.quote('http://localhost:8080')
client_id = 'YOUR-CLIENT-ID-HERE'

querystring = urlencode({
    'redirect_uri': callback_url,
    'client_id':    client_id,
    'grant_type':   'code'
})

redirect_to = f'https://www.googleapis.com?{querystring}'
\end{lstlisting}


Authorization and resource servers could be a third party service trusted by the
resource server.

\begin{itemize}
    \item  redirect uri, where the client will be redirect after authorization.
    \item  response type: type of the information the client expects to receive
        (most common code, (authorization code))
    \item  scope: permission that the client wants
    \item  consent: authorization server takes the scopes and verifies whether or not
        it want to give client permissions
    \item  client\_id: used to identify the client with the authorization server
    \item  client\_secret: secret password that only client and authorization server knows
    \item  \texttt{auth\_code}: short lived code that server sends to the client
    \item access\_token: key the client will use to communicate with the resource server.
\end{itemize}

\subsection{The protocol}
It all starts with a specifically crafted URL:
\begin{lstlisting}
https://www.googleapis.com
    ?redirect_uri=http://localhost:3030
    &client_id=
    &scope=email read-contacts
    &grant=code
\end{lstlisting}

In the login prompt, users sees the assertion that will be inserted in the
OAuth2 JWT.

OAuth2 flow is always between website A and B, the user has only to consent the
OAuth2 request.

It all starts with a specially crafted URL, besides the host, it includes extra
information that says what the host should do, they include the registered
application, the redirect uri (once discord is done the authorization the host
will send you there). Response type is for saying that you want a code back, and
scope is the actual data that you want to access.


\subsubsection{OAuth2}
Provide an access token at each client, which they could use to validate their
identity at each API request.  This token could be either valid, expired,
revoked or invalid.

\subsubsection{Small digression on API Key}
\begin{lstlisting}[language=bash]
$ curl --header 'Authorization: Apikey 1234567890abcdef'
\end{lstlisting}
API keys make sense when the users of an API are only developers.


\begin{lstlisting}[language=bash]
$ curl -v --header "Authorization: Bearer $ACCESS_TOKEN" \
    http://localhost:8080/api

> GET / HTTP/1.1
> Host: localhost:8080
> User-Agent: curl/7.72.0
> Accept: */*
> Authorization: Bearer ...
\end{lstlisting}

The OAuth2 protocol provides a secure and standardized way for creating and exchanging those token between client and server.


%==============================================================================
