
\section{OAuth2 Protocol}
OAuth2 (abbreviation of Open Authentication) is an authentication framework,
which enables applications to obtain limited access over protected resources,
that reside on an HTTP service.

It works by separating the concerns of resource provision and authentication to
two separate entities: the resource server and the authorization server.

Once the access is granted to a third party application by the authorization server,
an access token cryptographically signed is issued.

Since the token is signed or encrypted, the resource server, can blindly trust
the forged tokens, and checks if the required authorizations are contained in
the token.

\subsection{Client registration}
It's important to know that, before performing any authorization request, the
client should already know some credentials, that will be used to obtain an access token.
Those credentials are: the \textbf{client id} and \textbf{client secret}.

The means through which the client registers typically involve end-user
interaction with an HTML registration form, exposed in an amministrative
interface by the authorization server.
However there is no official method since it is beyond the OAuth2
specifications.

In the client registration, it is required to specify:
\begin{itemize}
    \item
        The client type, which dictates how the access token will be obtained
        (see page \pageref{sec:grant-types-overview}).

    \item The client redirect URL, which could be one or more.
    \item
        Any other information required by the authorization server, for example
        the application name, logo, website, description, etc.
\end{itemize}

\subsection{Protocol Overview}
\begin{figure}[h]
    \centering
    \begin{BVerbatim}
+--------+                               +---------------+
|        |--(A)- Authorization Request ->|   Resource    |
|        |                               |     Owner     |
|        |<-(B)-- Authorization Grant ---|               |
|        |                               +---------------+
|        |
|        |                               +---------------+
|        |--(C)-- Authorization Grant -->| Authorization |
| Client |                               |     Server    |
|        |<-(D)----- Access Token -------|               |
|        |                               +---------------+
|        |
|        |                               +---------------+
|        |--(E)----- Access Token ------>|    Resource   |
|        |                               |     Server    |
|        |<-(F)--- Protected Resource ---|               |
+--------+                               +---------------+
    \end{BVerbatim}
    \caption{OAuth2 flow overview \cite{ietf-oauth}}
    \label{fig:oauth-flow-overview}
\end{figure}

With that out of the way, right above (figure \ref{fig:oauth-flow-overview}),
you could see the main flow of the OAuth2 protocol.
\\
The final objective is always to obtain an access token, that will be used by
the client to perform authorized requests.

As we will see, this protocol is heavily reliant on HTTP over TLS
\cite{ietf-https} and HTTP redirects.


 \begin{itemize}
     \item[(A)]
         The client initiates the protocol by sending an authorization request,
         which is performed directly to the resource owner, or preferably via
         the authorization server as an intermediary.
     \item[(B)]
         The client receives an authorization grant, which in all shape or
         forms a credential, which represents the resource owner's
         authorization.  The authorization grant is expressed using one of the
         four grant types, defined in the specification, or using a custom
         specified grant type extension.
     \item[(C)]
         The client requests the access token, by authenticating with the
         authorization server and presenting the received authorization grant.
     \item[(D)]
         The authorization server authenticates the client and validates the
         received authorization grant, if valid the access token is returned.
     \item[(E)]
         The client requests the protected resource from the resource
         server providing the access token.
     \item[(F)]
         The resource server validates the access token.
         If the validation is successful, the requested resource is provided at the client,
         otherwise an authorization error is returned.
 \end{itemize}

\subsection{Different grants for different clients}
\label{sec:grant-types-overview}
The specification defines two types of client: \textbf{public} and \textbf{confidential}.

Public clients are the one that could not keep securely client credentials.
An example could be a mobile application: no matter how hard you try to hide
your client secret in a mobile application, an experienced programmer with a
just a disassembly tool could find and read the client secret.

On the other hand, confidential types of clients could securely store the
client secret, this category includes server side application, where the secret
could be securely stored in the back-end code and users have no easy way to
access it.

Since those client are extremely different, at least for a security point of
view, the specification defines different ways for obtaining the access token.

Each way of obtaining the access token is called \textbf{grant types}, and
there are four of them:
\textbf{grant types}.
\begin{itemize}
    \item
        \textbf{access code}: used for confidential client. (pag. \pageref{sec:auth-code})
    \item
        \textbf{implicit}: used for public clients. (pag. \pageref{sec:implicit})
    \item
        \textbf{client credentials}: used for confidential clients and does not
        require user interactions. (pag. \pageref{sec:client-credentials})
    \item
        \textbf{client password}: same as client credentials, it exists only
        for retro-compatibility purposes. (pag. \pageref{sec:grant-password})
\end{itemize}

\subsection{Access Token}
Access Tokens works as an abstraction over credentials, and are treated as such
by the resource server.
\\
They take the form of a string, that usually (but not always) is interpretable
from the resource and authorization servers.

Access tokens as you will expect, could be: value, expired, revoked or invalid.

Each access token is unique, and resource servers could verify it's validity and
look up the list of actions (called \textbf{scopes}), that have been granted
by the authorization server server to that specific access token.

The OAuth2 specification does not constrain a specific access token format:

\begin{center}
\textit{``Access tokens can have different formats, structures, and methods of
utilization''}.
\end{center}
\begin{flushright}
    - IETF \cite{ietf-oauth}
\end{flushright}

As we will see later, the OpenID Connect framework takes advantage of this
freedom, and imposes the JWT token format (see \ac{jwt} at page \pageref{jwt}).


