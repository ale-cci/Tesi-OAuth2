\section{OpenID connect}
\label{openid}
% TODO
As seen so far, we can describe OAuth2 as an extensible authorization
framework, which can provide an access token using the right grant type in the
right circumstance.

But what if in our authorized requests we want to know how performed them?
Out of the box, OAuth2 does not provide anything standard to work with that.

A simple solution would be to expose an endpoint to our authorization server,
in which we could retrieve user informations information by providing the
access token.

Since this is a fairly common problem, years ago, companies like Google and
Facebook found their own solution (see Facebook Connect), and to prevent the
creation of half a billion more protocols, OpenID Connect has been invented.

OpenID is a thin layer, built on top of OAuth2, that adds the ability to perform
the user authentication using the authorization provider when obtaining the access token.

This is done by returning an extra token, called \textbf{id token}, alongside
the access token, that contains basic informations about the user identity.

Here is a redirection example after a successful authentication:
\begin{lstlisting}
HTTP/1.1 302 Found
Location: https://client.example.org/cb
  ?code=SplxlOBeZQQYbYS6WxSbIA
  &id_token=eyJ0NiJ9[...]ZXso
  &state=af0ifjsldkj
\end{lstlisting}

\subsection{Obtaining the idToken}
The steps performed to obtain the id token are exactly the same as for obtaining the access token.
The only requirement is to add \texttt{openid} in the list of scopes and voilà! The response now contains the id token too.

To make a quick comparison with the OAuth2 standard request, here is how you obtain the id token. Feel free to compare it with the one at page \pageref{lst:authorization-code}:

\begin{lstlisting}
querystring = urlencode({
   'response_type': 'code',
   'client_id':     'YOUR-CLIENT-ID',
   'redirect_uri':  'http://localhost:8080',
   'scope':         'openid https://www.googleapis.com/auth/drive.file',
   'state':         '123-EXAMPLE'
})

redirect_to = (
   f'https://accounts.google.com/o/oauth2/v2/auth?{querystring}'
)
\end{lstlisting}

\subsection{Obtaining user information about the authenticated user}
Earlier i have said that id tokens are issued by the authorization server, and contain some basic information about the authorized user.
But how are these information stored? And how can we read them?

In figure \ref{fig:id_token} you could see the id token we will take as example
(i have added new line characters to improve the readability).

\begin{figure}[h]
\begin{verbatim}
eyJhbGciOiAiUlMyNTYiLCAia2lkIjogIjEiLCAidHlwIjogIkpXVCJ9.
eyJhdXRocyI6ICIiLCAiaXNzIjogImh0dHBzOi8vbG9jYWxob3N0OjgwMDAiLCAic3V
iIjogMiwgImlhdCI6IDE2MTQzNzI0MzUuMDM4MTg4NywgImV4cCI6IDE2MTQzNzYwMz
UuMDM4MTg4N30.
slCAkhLRu9w-rBhmLUh487UPrqkJF1cBpXE5LyWRR1KH1XcPA_QW_uen06p7eHI
GNKs6zSttlz0metnTjyuPFtkuU7I9Tu1djF2b5qVxOjbeDjGr_7ESuJZakKa7ljMloR
bEW65FRpTGllIPBmFOqp8VXlM1h30ogT_Mm-zl1DoSUcIIhrDT3qFEGvRz3nw049g2R
0hkTPrQK-J4bkG-7vf9f9H_PKOel2l2JtKk-3kZ-l8JNKpqM29BVpTRJzmuxNsZMwVP
JJtt-hqBinTFJ15YHwKf_hKT3bjybibmm2ciXjFHvK3p4HREdXwvsR7A4la4dto4FCt
V09IG1L9eF0kyjFkLdu2Unz7kf2YFz4fHvU7KPizJt3hPJASi_l8HJoBd1Y7sTPsjxf
IUycjgGp0yc7qwGl9ZuQmDXZ3dMj4VpBIEKGNMVbwU8IzInIFuC9R-NpqI961YCWxJ0
Qnly5rrHtw3EJy9GjW8u5cJkY7w0lJCbpvfULMqNpznSXKC0WaoictP7d80CKc9LwER
cyZY8kg4PMbZGhc2VHdEyGL4r1xZDqxZhdAYFMXRSs4DVJk5GkISyBmo2kE0rR3QYTA
NIuB80vbjuN9IzX6TTEPfFbvPdFd7oTrmN_xhR_uMu0Omv8f0o0bvXDeYRg1fL6AWFa
BOYO1qWCdTxj7r8as
\end{verbatim}
\caption{Example of id token}
\label{fig:id_token}
\end{figure}

If you have a trained eye, you will see that contains two dots, and it's
encoded in something that reminds \texttt{base64}.
In fact, those two dots divide the id token in three parts

\begin{enumerate}
    \item the \textbf{head}, which contains information about how the token has been created.
    \item the \textbf{body}, aka the content of the token
    \item and finally the \textbf{signature}, which is used to verify the token authenticity.
\end{enumerate}

Each section is encoded in \texttt{base64} and afterwards the padding is
removed, and each occurrence of the characters \texttt{\/} and \texttt{+}  are
replaced with \texttt{\_} and \texttt{-} respectively.

So, in order to decode the content of the id token we have to perform the two steps in reverse order

\begin{lstlisting}[label=lst:decode-jwt, caption=Decode JWT chunks]
import base64

def decode(chunk):
    # 1. replace the characters
    chunk_1 = chunk.replace('-', '+').replace('_', '/')

    # 2. Re-insert the padding
    chunk_2 = chunk_1 + '=='

    # 3. Decode
    return base64.b64decode(chunk_2)

if __name__ == '__main__':
    id_token = 'eyJhbGciOiAiUl[...]BOYO1qWCdTxj7r8as'
    head, body, signature = id_token.split('.')

    head_decoded = decode(head)
    body_decoded = decode(body)
    sig_decoded = decode(signature)
\end{lstlisting}


This specific way of packing informations takes the name of \acl{jwt}.
At page \pageref{jwt} we will see how to interpret the content of the head and how the signature is calculated.
For now let's take a look at the body:

\begin{lstlisting}
>>> import json
>>> json.loads(body_decoded)
{'auths': '',
 'iss': 'https://localhost:8000',
 'sub': 2,
 'iat': 1614372435.0381887,
 'exp': 1614376035.0381887}
\end{lstlisting}

Here we start to see some interesting fields:
\begin{itemize}
    \item \acs{iss}: \acl{iss}, who has created the JWT
    \item \texttt{sub}: Subscriber identity, uniquely identifies the owner of
        the JWT in the authentication server.
    \item \texttt{iat}: Issued at, unix timestamp of when was created
    \item \texttt{exp}: Expiry, unix timestamp of when the token will expire
\end{itemize}
The authorization server could any extra informations using custom fields, not
defined by the specification, like \texttt{auths}.

It is important to not re-invent the wheel.
OpenID specification defines a list of fields (called claims), most of them
optional, which could be included in the JWT (full list at page \pageref{sec:openid-claims}).
In this way, if you need to add to your JWT something like an email or a phone number,
you don't have to invent the object key name by yourself.
% TODO: are id-tokens different from authorization-tokens: yes
% TODO: https://www.c-sharpcorner.com/article/accesstoken-vs-id-token-vs-refresh-token-what-whywhen/

\subsection{Single Sign On and Identity Provider}
When an authorization server supports OIDC, it is referred as an
\textbf{Identity Provider}, since it provides informations about the owner,
back to the clients.

OpenID connect, enables scenarios where one login can be shared among multiple
applications. This is also known as single sign on.

For example an application can support single-sign-on (SSO) using social networking services (i.e.
Facebook or Twitter) so that a user can use a login that he already has and it's
comfortable using it, instead of creating another identity on the client application.

\subsection{Get more informations about authenticated users}
If you think that the informations provided in the id token are not enough for
your client application, OpenID specification defines a protected OAuth2
endpoint, called \textbf{userinfo}, which is used to return additional data
about the authenticated end-user.

To call this endpoint, the client makes a request using the access token,
obtained through the OpenID Connect authentication.

The response claims are normally represented by a JSON object that contains a
collection of name and value pair of each claim.

If the userinfo response is signed and/or encrypted, then the claims are
returned in a JWT and the content-type must be \texttt{application/jwt}.

Here is an example of userinfo request:
\begin{lstlisting}
requests.get('http://localhost:8000/userinfo', headers={
    'Authorization': f'Bearer {access_token}'
})
\end{lstlisting}
