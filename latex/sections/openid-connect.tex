\section{OpenID connect}
\label{openid}
% TODO
As seen so far, we can describe OAuth2 as an extensible authorization
framework, that can provide access token by using the right grant type in the
right circumstance.

But what if need to get some informations on the authenticated user who just
performed this request?
For example, what if we want to authenticate only using the authorization
server, and treat the access token as a "user session"?

A simple solution would be to expose an endpoint to our resource server, where
we could retrieve user informations information by providing the access token.

This is easily doable, but what if some day we want to switch to another identity provider?
The endpoint almost certainly be different, and we would need to adapt our
application.


To solve this problem, in the past companies like
Facebook  and Google found their own way to perform authentication using OAuth2
(see Facebook Connect).
To prevent the creation of half a billion protocols, the OpenID Connect
was created.

OpenID Connect (OIDC) is a thin layer build on top of OAuth2.0 that adds
authentication capabilities.


Alongside the access token, a JWT (more on JWTs at page \pageref{jwt}) is
returned, which takes the name of \textbf{id token}.

\subsection{Example of OpenID redirect uri}

\begin{lstlisting}
from urllib.parse import urlencode

querystring = urlencode({
   'response_type': 'code',
   'client_id':     'YOUR-CLIENT-ID',
   'redirect_uri':  'http://localhost:8080',
   'scope':         'openid https://www.googleapis.com/auth/drive.file',
   'state':         '123-EXAMPLE'
})

redirect_to = (
   f'https://accounts.google.com/o/oauth2/v2/auth?{querystring}'
)
\end{lstlisting}

Successful authentication response:
\begin{lstlisting}
HTTP/1.1 302 Found
Location: https://client.example.org/cb#
  code=SplxlOBeZQQYbYS6WxSbIA
  &id_token=eyJ0 ... NiJ9.eyJ1c ... I6IjIifX0.DeWt4Qu ... ZXso
  &state=af0ifjsldkj
\end{lstlisting}

Notice any difference with the one at page \pageref{lst:authorization-code}?
\\
OpenID requests are recognizable by the \texttt{openid} scope, included in the
scope lists.

% TODO: are id-tokens different from authorization-tokens: yes
% TODO: https://www.c-sharpcorner.com/article/accesstoken-vs-id-token-vs-refresh-token-what-whywhen/
\subsection{Single Sign On and Identity Provider}
OIDC establish a client to perform a login session.
When an authorization server supports OIDC it referred as an identity provider,
since it provides information about the resource owner back to the client.

OpenID connect, enables scenarios where one login can be shared among multiple
applications. It is also known as single sign on.
For example an application can support single-sign-on (SSO) with social networking services (i.e.
Facebook or Twitter) so that a user can use a login that he already has and it's
comfortable using it.

Example with ATM. ATM is the client and it communicates with the bank. The bank
card is the token issued by the bank. Not only gives access to the ATM to your
bank account, but holds some basic information about you, when the card expires
and why you are.
ATM cannot work without the infrastructure of the bank.

On top of OAuth2, a specific scope of OpenID is used. The authorization server
goes to all the steps listed before and issues the client an authorization code
via the client's browser.
The key difference is that the client receives both an access token and an id
token.

ID token is a specifically formatted token (JWT). Client can extract your user
id, your email, when you logged in, when it expires, and it can tell if anyone
has tried to tamper with the JWT.

Fields contained in the JWT are called claims.
There is also a standard way that a client can ask additional information from the
authorization server, such as the email address.

The full list of OpenID Connect specification reserved scopes is reported at
page \pageref{openid}.

\subsection{Get more informations about authenticated users}
OpenID specification defines a protected OAuth2 endpoint, that returns
additional informations (claims) about the authenticated end-user.
To obtain the requested claims, the client makes a request to the userinfo
endpoint, using an access token obtained through OpenID Connect
Authentication.

The response claims are normally represented by a JSON objet that contains a
collection of name and value pair of each claim.

If the UserInfo response is signed and/or encrypted, then the claims are
returned in a JWT and the content-type MUST be \texttt{application/jwt}.
The response may also be encrypted without being signed.


%==============================================================================
