\section{OpenID connect}
\label{openid}
OAuth2.0 is designed only for granting authorization to third party applications
to the resource server. Nothing more and nothing less.

To make an example, OAuth2 is like giving an application a key, that does not
tell anything about the owner's identity.

OpenID Connect (OIDC) is a thin layer that sits on top of OAuth2.0 that add functionality around
login and profile information about the person who is logged in.
Instead of a key, OpenID is like giving a person a badge. A badge not only gives
permission but also some basic information of who your are.

The OpenID requests in the OAuth2 protocol are easily recognizable by the
'openid' included in the scope list.

OIDC establish a client to perform a login session.
When an authorization server supports OIDC it referred as an identity provider,
since it provides information about the resource owner back to the client.

OpenID connect, enables scenarios where one login can be shared among multiple
applications. It is also known as single sign on.
For example an application can support single-sign-on (SSO) with social networking services (i.e.
Facebook or Twitter) so that a user can use a login that he already has and it's
comfortable using it.

Example with ATM. ATM is the client and it communicates with the bank. The bank
card is the token issued by the bank. Not only gives access to the ATM to your
bank account, but holds some basic information about you, when the card expires
and why you are.
ATM cannot work without the infrastructure of the bank.

On top of OAuth2, a specific scope of OpenID is used. The authorization server
goes to all the steps listed before and issues the client an authorization code
via the client's browser.
The key difference is that the client receives both an access token and an id
token.

ID token is a specifically formatted token (JWT). Client can extract your user
id, your email, when you logged in, when it expires, and it can tell if anyone
has tried to tamper with the JWT.

Fields contained in the JWT are called claims.
There is also a standard way that a client can ask additional information from the
authorization server, such as the email address.

The full list of OpenID Connect specification reserved scopes is reported at
page \pageref{openid}.

\subsection{the /userinfo endpoint}
OpenID specification defines a protected OAuth2 endpoint, that returns
additional informations (claims) about the authenticated end-user.
To obtain the requested claims, the client makes a request to the userinfo
endpoint, using an access token obtained through OpenID Connect
Authentication.

The response claims are normally represented by a JSON objet that contains a
collection of name and value pair of each claim.

If the UserInfo response is signed and/or encrypted, then the claims are
returned in a JWT and the content-type MUST be \texttt{application/jwt}.
The response may also be encrypted without being signed.


%==============================================================================
