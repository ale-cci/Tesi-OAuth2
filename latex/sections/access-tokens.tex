\section{Access tokens}
Since we are able to trust JWTs  without the use of a database, they offer the
main advantage to not issue database calls to verify the user identity.


Using JWTs instead of cookies for authentications grants the main advantage to
not issue a database, but as they sail "with great power comes great
responsibility", in fact if the token is stolen by someone, you cannot do
anything about it.

Since it is a completely stateless protocol, using JWT as is, there is no way
you can revoke access to the token. No database calls means that there is
nowhere to store that JWT is been revoked.

It is true that it is not easy to steal them, because OAuth2 protocol specifies
that all interactions between client and server are made using SSL protocol, but
it is not impossible.

One workaround to this problem is to make JWTs expire in a short lapse of time,
usually 15 minutes to an hour, then force clients either to re-authenticate or
provide them an extra token (\textit{refresh\_token}), used to re-claim
authentication on their behalf.

Refresh tokens are exactly like session: JWTs stored in the database, with a
long or usually without expiry date.
They should securely stored, in order to prevent \acs{xss} and \acs{csrf} attacks.

Even with refresh token we have the big advantage to query the database
approximately once each 15 minutes per user.

With that said, it still lies the problem that there is no easy way to revoke a
token. Even if a refresh token is deleted from the database, it's corrispettive
access token is still usable until expiry.

\subsection{Example of access token}
The following whitespaces between the access token sections are been introduced
only to simplify the readability.
It is important to say that Whitespaces characters are not allowed inside an
access token.

\begin{lstlisting}
eyJhbGciOiAiUlMyNTYiLCAia2lkIjogIjEiLCAidHlwIjogIkpXVCJ9.

eyJhdXRocyI6ICIiLCAiaXNzIjogImh0dHBzOi8vbG9jYWxob3N0OjgwMDAiLCAic3V
iIjogMiwgImlhdCI6IDE2MTQzNzI0MzUuMDM4MTg4NywgImV4cCI6IDE2MTQzNzYwMz
UuMDM4MTg4N30.

slCAkhLRu9w-rBhmLUh487UPrqkJF1cBpXE5LyWRR1KH1XcPA_QW_uen06p7eHI
GNKs6zSttlz0metnTjyuPFtkuU7I9Tu1djF2b5qVxOjbeDjGr_7ESuJZakKa7ljMloR
bEW65FRpTGllIPBmFOqp8VXlM1h30ogT_Mm-zl1DoSUcIIhrDT3qFEGvRz3nw049g2R
0hkTPrQK-J4bkG-7vf9f9H_PKOel2l2JtKk-3kZ-l8JNKpqM29BVpTRJzmuxNsZMwVP
JJtt-hqBinTFJ15YHwKf_hKT3bjybibmm2ciXjFHvK3p4HREdXwvsR7A4la4dto4FCt
V09IG1L9eF0kyjFkLdu2Unz7kf2YFz4fHvU7KPizJt3hPJASi_l8HJoBd1Y7sTPsjxf
IUycjgGp0yc7qwGl9ZuQmDXZ3dMj4VpBIEKGNMVbwU8IzInIFuC9R-NpqI961YCWxJ0
Qnly5rrHtw3EJy9GjW8u5cJkY7w0lJCbpvfULMqNpznSXKC0WaoictP7d80CKc9LwER
cyZY8kg4PMbZGhc2VHdEyGL4r1xZDqxZhdAYFMXRSs4DVJk5GkISyBmo2kE0rR3QYTA
NIuB80vbjuN9IzX6TTEPfFbvPdFd7oTrmN_xhR_uMu0Omv8f0o0bvXDeYRg1fL6AWFa
BOYO1qWCdTxj7r8as
\end{lstlisting}

\subsubsection{Head}
\begin{lstlisting}
eyJhbGciOiAiUlMyNTYiLCAia2lkIjogIjEiLCAidHlwIjogIkpXVCJ9.
\end{lstlisting}
\begin{lstlisting}
{"alg": "RS256", "kid": "1", "typ": "JWT"}
\end{lstlisting}

In the header is contained the information that explains how the JWT has been
created: the algorithm (\texttt{alg}), the type and the id of the encryption key
used (\texttt{kid}).
\\
If the JWT is of type JWS (it has a signature), usually the \texttt{kid} refers
to a public key, expsed in the jwks endpoint (more at page \pageref{jwks}).

\subsubsection{Body}
\begin{lstlisting}
eyJhdXRocyI6ICIiLCAiaXNzIjogImh0dHBzOi8vbG9jYWxob3N0OjgwMDAiLCAic3V
iIjogMiwgImlhdCI6IDE2MTQzNzI0MzUuMDM4MTg4NywgImV4cCI6IDE2MTQzNzYwMz
UuMDM4MTg4N30.
\end{lstlisting}
\begin{lstlisting}
{
    "auths": "",
    "iss": "https://localhost:8000",
    "sub": 2,
    "iat": 1614372435.0381887,
    "exp": 1614376035.0381887
}
\end{lstlisting}

OpenID specification describes these extra fields, required in the JWT body:
\begin{itemize}
    \item \texttt{sub}: subscriber identity, an unique identifier of the user in
        the authorization server.
    \item \texttt{iat} issued at, unix timestamp describing when the JWT has
        been created.
    \item \texttt{exp} expiry date of the JWT, expressed in unix timestamp.
    \item \texttt{auths} is an example of custom field inserted from the authorization
        provider
\end{itemize}

The full list of OpenID reserved claims could be found at page \pageref{openid}.

\subsubsection{Signature}
\begin{lstlisting}
slCAkhLRu9w-rBhmLUh487UPrqkJF1cBpXE5LyWRR1KH1XcPA_QW_uen06p7eHI
GNKs6zSttlz0metnTjyuPFtkuU7I9Tu1djF2b5qVxOjbeDjGr_7ESuJZakKa7ljMloR
bEW65FRpTGllIPBmFOqp8VXlM1h30ogT_Mm-zl1DoSUcIIhrDT3qFEGvRz3nw049g2R
0hkTPrQK-J4bkG-7vf9f9H_PKOel2l2JtKk-3kZ-l8JNKpqM29BVpTRJzmuxNsZMwVP
JJtt-hqBinTFJ15YHwKf_hKT3bjybibmm2ciXjFHvK3p4HREdXwvsR7A4la4dto4FCt
V09IG1L9eF0kyjFkLdu2Unz7kf2YFz4fHvU7KPizJt3hPJASi_l8HJoBd1Y7sTPsjxf
IUycjgGp0yc7qwGl9ZuQmDXZ3dMj4VpBIEKGNMVbwU8IzInIFuC9R-NpqI961YCWxJ0
Qnly5rrHtw3EJy9GjW8u5cJkY7w0lJCbpvfULMqNpznSXKC0WaoictP7d80CKc9LwER
cyZY8kg4PMbZGhc2VHdEyGL4r1xZDqxZhdAYFMXRSs4DVJk5GkISyBmo2kE0rR3QYTA
NIuB80vbjuN9IzX6TTEPfFbvPdFd7oTrmN_xhR_uMu0Omv8f0o0bvXDeYRg1fL6AWFa
BOYO1qWCdTxj7r8as
\end{lstlisting}

\begin{lstlisting}
# Signature verification
\end{lstlisting}

% \subsection{Access tokens in openid}
% TODO: Access token in openid example

\subsection{Pros and Cons of JWTs}
\subsubsection{Pros:}
\begin{itemize}
    \item Stateless
    \item great for api
    \item secure
    \item carry useful and trusted information
    \item can store information that can drive UX without backend interaction.
    \item No need for a centralized database.
\end{itemize}
\subsubsection{Cons:}
\begin{itemize}
    \item sharing secrets in a microservice architecture.
    \item Key management: where to put public key; how to kow if it's valid and
        deprecated keys.
    \item very tricky to consume correctly
    \item careful storage of refresh token
    \item Impossibility to revoke tokens
    \item Insecure implementation libraries (no algorithm)
\end{itemize}

