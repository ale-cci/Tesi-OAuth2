\section{OAuth2: grant type password}
This grant type is one of the simples grants, involving only one step: the
application presents a traditional username and password login form to collect
the user's credentials and make a POST request to the server to exchange the
password for an access token.

\begin{alltt}
POST /oauth/token HTTP/1.1
Host: authorization-server.com
Content-type: application/x-www-form-urlencoded

grant_type=password
&username=exampleuser
&password=1234luggage
&client_id=xxxxxxxxxx
\end{alltt}

The server replies with an access token, in the same format as the other grant
types.

\begin{lstlisting}
{
  "access_token": "MTQ0NjOkZmQ5OTM5NDE9ZTZjNGZmZjI3",
  "token_type": "bearer",
  "expires_in": 3600,
  "scope": "create"
}
\end{lstlisting}
\subsection{When the password grant type is used?}
Since this grant type requires to collect the user's password, this is the exact
problem that OAuth was created to avoid in the first place, so why was it
included as part of OAuth in first place?

While a service should never let a third party app use the password grant, it is
reasonable for a service's own application to ask the user to enter their
password.
For example, if you download Twitter's mobile app, you would not be surprised if
the first thing it does when launched is ask for you Twitter password.
On the other hand if you download a third-party Gmail application, it should use
Google's OAuth server rather than ask you to enter your Gmail password.

This grant type let's the application take advantage of the rest of benefits
that OAuth provides around access tokens and token lifetimes.
The user's password instead of being stored on the device, is exchanged with the
access token.

