\section{OAuth2: implicit flow}
\label{sec:implicit}
Public applications, such as desktop and mobile apps, are not able to store
securely store client credentials, in particular the client secret.

For this particular reason the OAuth2 specification felt the urge to alter the
main protocol, and completely skip the step of exchanging the authorization code
and client secret with the access token.
Instead the access token is returned immediately after the owner grants the
permissions, as a GET parameter in the redirect URL.

\begin{figure}[h]
    \centering
    \begin{BVerbatim}
+----------+
| Resource |
|  Owner   |
|          |
+----------+
     ^
     |
    (B)
+----|-----+          Client Identifier     +---------------+
|         -+----(A)-- & Redirection URI --->|               |
|  User-   |                                | Authorization |
|  Agent  -|----(B)-- User authenticates -->|     Server    |
|          |                                |               |
|          |<---(C)--- Redirection URI ----<|               |
|          |          with Access Token     +---------------+
|          |            in Fragment
|          |                                +---------------+
|          |----(D)--- Redirection URI ---->|   Web-Hosted  |
|          |          without Fragment      |     Client    |
|          |                                |    Resource   |
|     (F)  |<---(E)------- Script ---------<|               |
|          |                                +---------------+
+-|--------+
  |    |
 (A)  (G) Access Token
  |    |
  ^    v
+---------+
|         |
|  Client |
|         |
+---------+
\end{BVerbatim}
    \caption{Implicit grant flow \cite{ietf-oauth}}
    \label{fig:authorization-implicit-grant-flow}
\end{figure}

\begin{enumerate}
    \item[(A)]
        The client application redirect to the authorization server, (always
        providing the client id, callback url and scopes)
    \item[(B)]
        The users authenticates and chooses to grant or deny the
        authorization.
    \item[(C)]
        The authorization server issues a redirect to the callback URL,
        containing the access token as GET parameter.

    \item[(D)]
        The user agent follows the redirect, exposing to the client application the access token.
    \item[(E)]
        The clients provides the javascript code, used to extract the access token and clean up the
        URL parameters to the user agent.
        \\
        This way, once it is executed, no other application could read the access token anymore.
    \item[(F)]
        The user agent executes the provided script, and extracts the access token.
    \item [(G)]
        User agent provides the access token to the client application.
\end{enumerate}


This protocol is a bit "faster" than the authorization code grant (page
\pageref{sec:auth-code}) (if you want to define it like that), since it skips
the intermediate step to exchange code with access
token.

But it is clear that if an external application (like a chrome extension) reads
your "\texttt{window.location}", you are basically done, and the "innocent" chrome
extension could perform OAuth2 requests with your account, without you having to
give it the permissions!

A solution to this problem is discussed later in this dissertation at the PKCE
section (pg. \pageref{pkce})

It is advised to remove the access token from the URL once you have read it, to
minimize the possibility to leak it.

\subsection{Two channel communications: trusted and insecure}
There are no perfectly secured channels, but there are at
best, only ways to make a less secure channel more secure.

In the standard flow of the OAuth2 protocol (grant code), two channels are
involved:
\begin{itemize}
    \item a front-end channel, used for redirect and a first exchange of client
        id with an authorization code
    \item a back-end channel, used for exchanging the code received from the
        less trusted channel with the authorization provider
\end{itemize}

The "back-end channel" takes the name of "trusted channel". On the other hand,
the fronted takes the names of "insecure channels", because it's easily tampered
and overheard.

\subsection{PKCE (Proof Key for Code Exchange)}
\label{pkce}
Public applications, such as desktop and mobile apps, are unable to securely
store credentials, in our case, the client secret.

For this particular reason the implicit flow has been created, so
that the step after the authorization that exchanges the code and the secret
with the authorization code is skipped, so the authorization code is immediately
returned.

This particular sequence of events opens up a vulnerability:
when the access token is returned by the authorization response, malicious apps
running alongside our OAuth2 client are able to read it, and therefore use it as
our application would.

For example, if your client is running in a browser, let's say firefox just to
pick one, any plugin you have installed has access to the current url, and
therefore to the access\_token.

\ac{pkce} is a superset feature for OAuth2, idealized for preventing this kind of
attack, called "authorization code interception".

OAuth2.1 specification will probably mandate that all implementation of the authorization
grant type use \ac{pkce}.

\begin{enumerate}
    \item the client generates a random string, and calculates the hash using a
        hashing function
    \item the client performs the normal oauth2 flow, providing along with the
        client secret, the random string hashed value.
    \item when the authorization server performs a successful redirect with the
        'code', the client exchanges with a post request, the code and the
        unhashed secret with the access token.
\end{enumerate}

This feature blocks "easy" oauth2 attacks, that could be performed simply by
reading the access token from the GET request.

