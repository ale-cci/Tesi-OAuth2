\section{OAuth2: implicit flow}
\label{sec:implicit}
Since public clients could not store a "client secret" to exchange for an
authorization code, the protocol is slightly altered.
Instead of returning the authorization code, the access token is returned
directly in the GET parameters of the redirect.

Therefore the steps required to perform the authorization are reduced to 3:
\begin{enumerate}
    \item Redirect to the authorization provider
    \item User performs the authentication and approves the requested
        permissions
    \item The access token is returned to the third party client.
\end{enumerate}

\begin{figure}[h]
    \centering
    \begin{BVerbatim}
+----------+
| Resource |
|  Owner   |
|          |
+----------+
     ^
     |
    (B)
+----|-----+          Client Identifier     +---------------+
|         -+----(A)-- & Redirection URI --->|               |
|  User-   |                                | Authorization |
|  Agent  -|----(B)-- User authenticates -->|     Server    |
|          |                                |               |
|          |<---(C)--- Redirection URI ----<|               |
|          |          with Access Token     +---------------+
|          |            in Fragment
|          |                                +---------------+
|          |----(D)--- Redirection URI ---->|   Web-Hosted  |
|          |          without Fragment      |     Client    |
|          |                                |    Resource   |
|     (F)  |<---(E)------- Script ---------<|               |
|          |                                +---------------+
+-|--------+
  |    |
 (A)  (G) Access Token
  |    |
  ^    v
+---------+
|         |
|  Client |
|         |
+---------+
\end{BVerbatim}
    \caption{Implicit grant flow \cite{ietf-oauth}}
    \label{fig:authorization-implicit-grant-flow}
\end{figure}


This protocol is a bit "faster" than the authorization code grant (page \pageref{sec:auth-code}) (if you want to define it
like that), because it skips the intermediate step to exchange code with access
token.

But it is clear that if an external application (like a chrome extension) reads
your "\texttt{window.location}", you are basically done, and the innocent chrome
extension could perform OAuth2 requests with your account, without you having to
give it the permissions!

A solution to this problem is discussed later in this dissertation at the PKCE
section (pg. \pageref{pkce})

It is advised to remove the access token from the URL once you have read it, to
minimize the possibility to leak it.

\subsection{Two channel communications: trusted and insecure}
There are no perfectly secured channels in the real world, but there are at
best, only ways to make a less secure channel more secure.

In the standard flow of the OAuth2 protocol (grant code), two channels are
involved:
\begin{itemize}
    \item a front-end channel, used for redirect and a first exchange of client
        id with an authorization code
    \item a back-end channel, used for exchanging the code received from the
        less trusted channel with the authorization provider
\end{itemize}

The "back-end channel" takes the name of "trusted channel". On the other hand,
the fronted takes the names of "insecure channels", because it's easily tampered
and overheard.

%==============================================================================
