% vim: tw=80
\section{OAuth2: authorization code}
\label{sec:auth-code}
\begin{figure}[h]
    \centering
    \begin{BVerbatim}
+----------+
| Resource |
|   Owner  |
|          |
+----------+
     ^
     |
    (B)
+----|-----+          Client Identifier      +---------------+
|         -+----(A)-- & Redirection URI ---->|               |
|  User-   |                                 | Authorization |
|  Agent  -+----(B)-- User authenticates --->|     Server    |
|          |                                 |               |
|         -+----(C)-- Authorization Code ---<|               |
+-|----|---+                                 +---------------+
  |    |                                         ^      v
 (A)  (C)                                        |      |
  |    |                                         |      |
  ^    v                                         |      |
+---------+                                      |      |
|         |>---(D)-- Authorization Code ---------'      |
|  Client |          & Redirection URI                  |
|         |                                             |
|         |<---(E)----- Access Token -------------------'
+---------+       (w/ Optional Refresh Token)
    \end{BVerbatim}
    \caption{Authorization code grant flow \cite{ietf-oauth}}
    \label{fig:authorization-code-grant-flow}
\end{figure}
Note: The lines illustrating steps (A), (B), and (C) are broken into
two parts as they pass through the user-agent.

As defined in the specification, this is the most secure way to obtain an access token.
It uses at it's best two channels of communication:
A front-end channel, less secure but with user interaction, and a back-end channel, more
secure but that lives on the server side of the client application.

If you don't know what a \textbf{user-agent} is, you could think of it for now
as the internet browser used by the user, ex: Firefor or Chrome.

\subsection{Steps required to obtain the access token}
Let's see together what are the steps required to obtain an access token:
\begin{enumerate}
    \item[(A)]
        The authorization process starts with the client, redirecting the owner's user agent
        to the authorization server.

        In the redirect URL, the client identifies itself by providing the
        client id and a callback URL as GET parameters, and specifies the
        permissions that he wants to access.

        The crafted URL points to the 'authorization page' of the authorization server.

    \item[(B)]
        The authorization server, authenticates the resource owner, and asks him to allow or
        deny, the client's requested permissions.

    \item[(C)]
        Assuming that the permission's request was approved, the authorization server
        redirects the user-agent back to the client, using the callback URL provided earlier.

        In this redirect is included as GET parameter a token called "authorization code".

    \item[(C)]
        If the permission request was denied, the authorization server performs
        a 302 redirect to the client application, providing a GET parameter
        named \texttt{error}, containing a short message of what went wrong.

    \item[(D)]
        The client application reads the authorization code, and performs a post request
        to the authorization server to obtain the access token, providing as parameters:
        the \texttt{authorization\_code}, the client id and secret, the callback
        uri (\texttt{redirect\_uri}), and the grant type.

        Since this is the authorization code grant type, this value will always
        be equal to \texttt{code}.


        \begin{verbatim}
    POST /oauth/token HTTP/1.1
    Host: authorization-server.com

    grant_type=authorization_code
    &code=xxxxxxxxxxx
    &client_id=xxxxxxxxxx
    &client_secret=xxxxxxxxxx
    &redirect_uri=https://example-app.com/redirect
        \end{verbatim}


    \item[(E)]
        The authorization sever authenticates the client, validates the
        authorization code, and ensures that the redirect URI received, matches
        the URI used to redirect the client in step (C).

        If no problem occurs, the authorization serve answers back with an
        access token, and optionally a refresh token (see page
        \pageref{sec:refresh-token})
\end{enumerate}

Authorization codes are unique, when an authorization request is performed, a
new authorization code is specifically created for that request.

If an authorization code is used more than once, the authorization request must
deny the subsequent requests.


\begin{figure}[h!]
    \begin{lstlisting}[label={lst:code}, caption={Obtain the access token with grant type authorization code}]
#!/usr/bin/env python3
from urllib.parse import urlencode

# [A] Performing user redirect
querystring = urlencode({
    'response_type': 'code',
    'client_id':     'YOUR-CLIENT-ID',
    'redirect_uri':  'http://localhost:8080',
    'scope':         'https://www.googleapis.com/auth/drive.file',
    'state':         '123-EXAMPLE'
})

redirect_to = (
    f'https://accounts.google.com/o/oauth2/v2/auth?{querystring}'
)

print(f'Open in your browser: {redirect_to}')


# [A] Wait for for user-agent redirect
import http.server
from urllib.parse import urlparse, parse_qs

handler = http.server.SimpleHTTPRequestHandler

print('Waiting for authorization...')
with http.server.HTTPServer(('', port), handler) as server:
    sock, _ = server.get_request()
    raw_http = sock.recv(2048).decode('utf-8')

raw_url, _ = raw_http.split('\r\n', 1)
request_url = urlparse(raw_url)

params = parse_qs(request_url.query)
auth_code = params['code'][0]


# [D] Exchange authorization code with access token:
import os, requests, pprint

print(f'Authorization code: {auth_code}')
response = requests.post('https://oauth2.googleapis.com/token', json={
    'code':          auth_code,
    'client_id':     os.environ['YOUR-CLIENT-ID'],
    'client_secret': os.environ['YOUR-CLIENT-SECRET'],
    'redirect_uri':  f'http://localhost:{port}',
    'grant_type':    'authorization_code'
})

jresp = response.json()
access_token = jresp['access_token']
pprint.pprint(jresp)

\end{lstlisting}
\end{figure}

\subsection{Revoke an access token}
By requirements, an OAuth2 authentication server must expose an endpoint for
revoking the validity of a particular token.
