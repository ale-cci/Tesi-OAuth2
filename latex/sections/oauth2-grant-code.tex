\section{OAuth2: authorization code}
\label{sec:auth-code}
\begin{figure}[h]
    \centering
    \begin{BVerbatim}
+----------+
| Resource |
|   Owner  |
|          |
+----------+
     ^
     |
    (B)
+----|-----+          Client Identifier      +---------------+
|         -+----(A)-- & Redirection URI ---->|               |
|  User-   |                                 | Authorization |
|  Agent  -+----(B)-- User authenticates --->|     Server    |
|          |                                 |               |
|         -+----(C)-- Authorization Code ---<|               |
+-|----|---+                                 +---------------+
  |    |                                         ^      v
 (A)  (C)                                        |      |
  |    |                                         |      |
  ^    v                                         |      |
+---------+                                      |      |
|         |>---(D)-- Authorization Code ---------'      |
|  Client |          & Redirection URI                  |
|         |                                             |
|         |<---(E)----- Access Token -------------------'
+---------+       (w/ Optional Refresh Token)
    \end{BVerbatim}
    \caption{Authorization code grant flow \cite{ietf-oauth}}
    \label{fig:authorization-code-grant-flow}
\end{figure}
Note: The lines illustrating steps (A), (B), and (C) are broken into
two parts as they pass through the user-agent.


The authorization code grant type is used to obtain both access and refresh
tokens, and is optimized for confidential clients.
Since it is a redirection-based flow, the clients must be capable of interacting
with the resource owner's user-agent, and capable of receiving incoming requests
(via redirection) from the authorization server.

\subsection{Steps required to obtain the access token}
% NOTE: re-read and explain in more human words this
\begin{enumerate}
    \item[(A)] the client redirects the resource owner's user agent to the
        authorization endpoint, and provides the client identifier, request
        scope state and redirection URI, to which the authorization server will
        send the user agent back, once the access is granted (or denied).
    \item[(B)] the authorization server authenticates the resource owner (via
        user agent) and asks the resource owner to allow or deny, the client's
        access request.
    \item[(C)] Assuming the resource's access is granted, the authorization
        server redirects the user-agent back to the client using the redirection
        URI provided earlier.
    \item[(D)] The client exchanges the received authorization code, client id,
        client secret and once again the redirection uri for the authorization code.

    \item[(E)] the authorization server authenticates the client, validates the
        authorization code, and ensures that the redirection URI received
        matches the URI used to redirect the client in step (C). If valid, the
        authorization server responds back with an access token and,
        optionally a refresh token.
\end{enumerate}

\begin{enumerate}
    \item
        Redirect to the authorization provider.
        \begin{alltt}
    http://google.apis.com?response_type=code\&client_id...
        \end{alltt}

    \item
        If the authorization is successful, the OAuth2 server will redirect to
        \lstinline{redirect\_uri} passing a \lstinline{code}
        as GET parameter. Otherwise \lstinline{error} contains the reason why the authorization was not successful.
        \begin{alltt}
    POST /oauth/token HTTP/1.1
    Host: authorization-server.com

    grant_type=authorization_code
    &code=xxxxxxxxxxx
    &redirect_uri=https://example-app.com/redirect
    &client_id=xxxxxxxxxx
    &client_secret=xxxxxxxxxx
        \end{alltt}

        \textit{If an authorization code is used more than once, the authorization server must deny the subsequent request.}

    \item
        The client exchanges the \lstinline{code}, called also "grant token", with the server to obtain an
        \\
        \lstinline{access\_token}.
    \item The server returns the access token with additional informations, such
        as expire date and JWT.
    \item
        The client uses the access token for APIs requests.
\end{enumerate}

\subsection{Revoke an access token}
By requirements, an OAuth2 authentication server must expose an endpoint for
revoking the validity of a particular token.

In case the authentication is performed via JWT (see page \pageref{jwt})
Something something

\subsection{The protocol}

It starts with a specifically crafted URL, where the user agent will be
redirected when the client needs to access the protected resources:

\begin{lstlisting}
https://www.googleapis.com
    ?redirect_uri=http://localhost:3030
    &client_id=
    &scope=email read-contacts
    &grant=code
\end{lstlisting}

Besides the host, it includes extra
information that says what the host should do, they include the registered
application, the redirect uri (once discord is done the authorization the host
will send you there). Response type is for saying that you want a code back, and
scope is the actual data that you want to access.

The URL contains a query string
It starts with a specifically crafted redirect URL, that the clients
redirects the user to.
The URL points to the authorization's server, and it contains parameters in the
form of a query string. Here is an example:

\begin{lstlisting}
import urllib.request

callback_url = urllib.request.quote('http://localhost:8080')
client_id = 'YOUR-CLIENT-ID-HERE'

querystring = urlencode({
    'redirect_uri': callback_url,
    'client_id':    client_id,
    'grant_type':   'code'
})

redirect_to = f'https://www.googleapis.com?{querystring}'
\end{lstlisting}


\begin{itemize}
    \item  redirect uri, where the client will be redirect after authorization.
    \item  response type: type of the information the client expects to receive
        (most common code, (authorization code))
    \item  scope: permission that the client wants
    \item  consent: authorization server takes the scopes and verifies whether or not
        it want to give client permissions
    \item  client\_id: used to identify the client with the authorization server
    \item  client\_secret: secret password that only client and authorization server knows
    \item  \texttt{auth\_code}: short lived code that server sends to the client
    \item access\_token: key the client will use to communicate with the resource server.
\end{itemize}

In the login prompt, users sees the assertion that will be inserted in the
OAuth2 JWT.
